\abstract{
Identifying appropriate research collaborators is a critical, challenging task in the scientific research community.
Traditional methods rely only on social networks and the number of author citations, which can be limited in scope and efficiency.
This thesis explores a hybrid \gls{ai}-driven approach that combines \glspl{kg} and \glspl{llm} to improve the accuracy and interpretability of collaborator recommendations.
To enhance retrieval and contextual relevance, the system employs a \gls{rag} approach, which dynamically retrieves relevant information from structured knowledge sources before generating recommendations.
By leveraging the structured semantic relationships of \glspl{kg} and the natural language understanding capabilities of \glspl{llm}, the system aims to provide contextual and explainable recommendations tailored to researchers' areas of expertise and project relevance.
To evaluate the effectiveness of different \glspl{llm} in \gls{rag} tasks, various models were tested.
The results indicate that Claude Sonnet 3.5 outperforms other \glspl{llm} in \gls{rag}-based recommendation metrics, demonstrating superior retrieval quality, contextual reasoning, and reduced hallucinations.
This work contributes to the advancement of the \gls{ai}-assisted research network, with the goal of increasing opportunities for collaboration and facilitating interdisciplinary research connections in a scalable and automated way.
}