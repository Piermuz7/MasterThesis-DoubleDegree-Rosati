\gls{bems} are critical in the effort to reduce energy consumption and enhance sustainability in residential, commercial, and industrial buildings. \gls{bems} integrate various technologies to monitor, control, and optimize the energy usage within buildings, ensuring efficient operation and reduced environmental impact. This literature review examines the evolution, methodologies, technologies, applications, challenges, and future directions of \gls{bems}.
insert somewhere \cite{manic2016building}

\subsection*{Historical Context and Evolution}

\subsubsection*{Early Developments}

The concept of managing building energy consumption dates back to the 1970s, spurred by the oil crises that highlighted the need for energy conservation. Initial efforts focused on manual control systems and basic automation to manage heating, ventilation, and air conditioning (HVAC) systems. These early systems were often limited in scope and functionality, primarily due to technological constraints.

\subsubsection*{Advancements in Automation and Control}

The 1980s and 1990s saw significant advancements in automation and control technologies, enabling more sophisticated \gls{bems}. The integration of microprocessors and the development of control algorithms allowed for more precise control of building systems. During this period, building automation systems (BAS) began to emerge, providing centralized control over HVAC, lighting, and other building systems.

\subsubsection*{The Digital Age and Smart Buildings}

The turn of the 21st century brought about the digital age, which revolutionized \gls{bems}. The advent of the Internet of Things (IoT), advanced sensors, and big data analytics enabled real-time monitoring and control of building energy systems. Smart buildings, equipped with intelligent \gls{bems}, became capable of autonomously adjusting energy usage based on occupancy, weather conditions, and other factors.

\subsection*{Key Components and Methodologies}

\subsubsection*{Sensors and Data Acquisition}

Sensors are fundamental to \gls{bems}, providing the necessary data for monitoring and control. These include temperature sensors, humidity sensors, occupancy sensors, and energy meters. Data acquisition systems collect and transmit sensor data to the \gls{bems} for processing.

Advantages:
\begin{itemize}
    \item Real-time monitoring of various parameters.
    \item Enhanced data accuracy and reliability.
\end{itemize}

Challenges:
\begin{itemize}
    \item Integration of heterogeneous sensor types.
    \item Managing and processing large volumes of data.
\end{itemize}

\subsubsection*{Control Systems}

Control systems are the core of \gls{bems}, responsible for regulating building systems based on sensor inputs and predefined algorithms. These systems use various control strategies, including:

\begin{itemize}
    \item Proportional-Integral-Derivative (PID) Control: A feedback control loop mechanism widely used in industrial control systems.
    \item Model Predictive Control (MPC): An advanced control strategy that uses a model of the system to predict future states and optimize control actions.
\end{itemize}

Advantages:
\begin{itemize}
    \item Improved energy efficiency through precise control.
    \item Ability to adapt to changing conditions.
\end{itemize}

Challenges:
\begin{itemize}
    \item Complexity in modeling and implementing control algorithms.
    \item Ensuring robustness and reliability in real-world conditions.
\end{itemize}

\subsubsection*{Communication Networks}

Effective communication networks are essential for the seamless operation of BEMS. These networks facilitate data exchange between sensors, control systems, and management platforms. Common communication protocols include BACnet, Modbus, and LonWorks.

Advantages:
\begin{itemize}
    \item Enhanced interoperability between different systems and devices.
    \item Real-time data transmission and control.
\end{itemize}

Challenges:
\begin{itemize}
    \item Ensuring cybersecurity and data privacy.
    \item Integrating legacy systems with modern communication protocols.
\end{itemize}

\subsubsection*{Data Analytics and Machine Learning}

Data analytics and machine learning play a crucial role in modern BEMS, enabling predictive maintenance, anomaly detection, and optimization of energy usage. Techniques such as regression analysis, clustering, and neural networks are commonly used.

Advantages:
\begin{itemize}
    \item Enhanced predictive capabilities and decision-making.
    \item Continuous improvement through learning and adaptation.
\end{itemize}

Challenges:
\begin{itemize}
    \item Handling large datasets and ensuring data quality.
    \item Developing accurate and generalizable models.
\end{itemize}

\subsection*{Applications of BEMS}

\subsubsection*{Energy Optimization}

\gls{bems} are primarily used to optimize energy consumption in buildings, reducing operational costs and environmental impact. Techniques such as demand response, load shifting, and peak load management are employed to achieve these goals.

Examples:
\begin{itemize}
    \item Demand Response: Adjusting energy usage based on real-time electricity prices or grid demand.
    \item Load Shifting: Moving energy-intensive tasks to off-peak hours to reduce demand charges.
\end{itemize}

\subsubsection*{HVAC Control}

HVAC systems are significant energy consumers in buildings. \gls{bems} optimize HVAC operations by adjusting temperature settings, controlling airflow, and managing equipment schedules based on occupancy and weather forecasts.

Examples:
\begin{itemize}
    \item Temperature Control: Maintaining optimal indoor temperatures while minimizing energy use.
    \item Ventilation Management: Regulating airflow to ensure indoor air quality and energy efficiency.
\end{itemize}

\subsubsection*{Lighting Control}

\gls{bems} improve lighting efficiency by using occupancy sensors, daylight harvesting, and automated scheduling. These systems ensure that lights are used only when needed, reducing energy waste.

Examples:
\begin{itemize}
    \item Occupancy Sensors: Automatically turning off lights in unoccupied areas.
    \item Daylight Harvesting: Adjusting artificial lighting based on the availability of natural light.
\end{itemize}

\subsubsection*{Renewable Energy Integration}

\gls{bems} facilitate the integration of renewable energy sources, such as solar panels and wind turbines, into building energy systems. They manage the generation, storage, and consumption of renewable energy, maximizing its use and reducing reliance on the grid.

Examples:
\begin{itemize}
    \item Solar Energy Management: Optimizing the use of solar power through real-time monitoring and control.
    \item Energy Storage: Managing battery storage systems to balance supply and demand.
\end{itemize}

\subsubsection*{Building Performance Monitoring}

\gls{bems} continuously monitor building performance, providing insights into energy consumption patterns, equipment performance, and potential areas for improvement. This data is used to identify inefficiencies and implement corrective actions.

Examples:
\begin{itemize}
    \item Energy Audits: Conducting regular audits to assess energy performance and identify savings opportunities.
    \item Fault Detection: Identifying and diagnosing equipment faults to ensure optimal operation.
\end{itemize}


\subsection*{Challenges in \gls{bems}}

\subsubsection*{Integration of Diverse Systems}

One of the primary challenges in \gls{bems} is integrating diverse systems and devices, often from different manufacturers, into a cohesive management platform. This requires standardized communication protocols and interoperability.

Solutions:
\begin{itemize}
    \item Developing and adopting open standards for communication and data exchange.
    \item Using middleware solutions to bridge compatibility gaps.
\end{itemize}

\subsubsection*{Cybersecurity and Data Privacy}

As \gls{bems} become more connected and data-driven, ensuring cybersecurity and data privacy becomes critical. Unauthorized access to \gls{bems} can lead to operational disruptions, data breaches, and even physical damage.

Solutions:
\begin{itemize}
    \item Implementing robust encryption and authentication mechanisms.
    \item Regularly updating and patching software to address vulnerabilities.
\end{itemize}

\subsubsection*{Scalability and Flexibility}

\gls{bems} must be scalable and flexible to accommodate changing building requirements and technological advancements. This involves designing systems that can easily integrate new devices and functionalities.

Solutions:
\begin{itemize}
    \item Using modular architectures that allow for incremental upgrades.
    \item Leveraging cloud-based platforms for scalable data storage and processing.
\end{itemize}

\subsubsection*{Cost and Return on Investment}

The initial cost of implementing \gls{bems} can be high, posing a barrier for adoption. Demonstrating a clear return on investment (ROI) is essential to justify the costs.

Solutions:
\begin{itemize}
    \item Conducting detailed cost-benefit analyses to highlight long-term savings.
    \item Providing financing options and incentives for energy-efficient upgrades.
\end{itemize}

\subsection*{Future Directions}

\subsubsection*{Advanced Analytics and Machine Learning}

The future of \gls{bems} lies in the continued development and application of advanced analytics and machine learning techniques. These technologies will enable more accurate predictions, adaptive control strategies, and automated fault detection.

Examples:
\begin{itemize}
    \item Predictive Maintenance: Using machine learning to predict equipment failures and schedule maintenance proactively.
    \item Energy Forecasting: Leveraging advanced analytics to predict energy consumption and optimize operations.
\end{itemize}

\subsubsection*{Internet of Things (IoT) and Edge Computing}

The proliferation of IoT devices and the advent of edge computing will transform \gls{bems} by enabling real-time data processing and control at the edge of the network. This will enhance responsiveness and reduce latency.

Examples:
\begin{itemize}
    \item Edge Analytics: Performing data analysis and decision-making at the edge, closer to the source of data.
    \item IoT Integration: Connecting a wide range of sensors and devices to create a more interconnected and responsive \gls{bems}.
\end{itemize}

\subsubsection*{Blockchain for Energy Management}

Blockchain technology holds potential for enhancing transparency, security, and efficiency in energy management. It can facilitate peer-to-peer energy trading, secure data sharing, and decentralized control.

Examples:
\begin{itemize}
    \item Peer-to-Peer Energy Trading: Enabling buildings to trade excess energy directly with each other using blockchain.
    \item Secure Data Sharing: Using blockchain to ensure the integrity and security of energy data.
\end{itemize}

\subsubsection*{Human-Centric Building Management}

Future \gls{bems} will increasingly focus on human-centric building management, prioritizing occupant comfort and well-being. This involves integrating indoor environmental quality (IEQ) metrics and user preferences into energy management strategies.

Examples:
\begin{itemize}
    \item Occupant Feedback Systems: Using feedback from occupants to adjust building systems for optimal comfort.
    \item Personalized Climate Control: Implementing systems that tailor the indoor environment to individual preferences.
\end{itemize}

\subsection*{Conclusion}
\acrlong{bems} are essential for achieving energy efficiency and sustainability in modern buildings. They integrate various technologies to monitor, control, and optimize energy usage, providing significant benefits in terms of cost savings and environmental impact. Despite the challenges, advancements in analytics, IoT, blockchain, and human-centric design offer promising directions for the future of \gls{bems}. Continued research and innovation will drive the evolution of these systems, making them more intelligent, responsive, and sustainable.