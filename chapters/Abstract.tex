\abstract{
The success of research project proposals heavily depends on the consortium, which should be experienced and knowledgeable in the topics outlined in the corresponding calls, e.g., those in the last EU's research and innovation programme Horizon Europe.
Yet, one of the most challenging activities is the formation of the consortium, which requires the identification of adequate research collaborators.
Traditional methods take this challenge by relying solely on social networks and, or the number of author citations, which proved to be limited in efficacy.
This thesis proposes an Agentic Graph \gls{rag} method, that provides contextual and explainable recommendations, which are tailored to researchers' areas of expertise and project relevance, thus more effective than existing approaches.
The proposed method combines \glspl{kg} and \glspl{llm} capabilities and has been developed following the Design Science research methodology.
The new method has been evaluated by considering one of the highest performant LLMs currently in the market, GPT-4o.
}