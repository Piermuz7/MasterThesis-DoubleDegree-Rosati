Data collection is the systematic process of gathering and measuring information on specific variables to answer relevant questions and assess outcomes.
According to \textcite{SaundersMark2023}, there are four primary methods of data collection:
\begin{itemize}
    \item Questionnaires: these are an efficient way to collect data as they consist of fixed-response questions that can be easily analyzed and compared. However, a key limitation is that participants cannot clarify their answers, which may lead to misinterpretations.
    \item Interviews: interviews can be structured like questionnaires, semi-structured with a mix of predefined and open-ended questions, or unstructured, allowing the interviewer to guide the conversation. While interviews are typically conducted one-on-one, they can also be held in group settings. A major challenge is finding suitable interview participants, and there is always a risk of bias. Additionally, conducting multiple interviews can be time-consuming.
    \item Observation: in observational research, a researcher studies participants' behavior to identify patterns. While this method provides real-world insights, interpreting observations correctly and selecting appropriate subjects for observation can be challenging.
    \item Archival Research: this method involves analyzing historical documents and textual materials from organizations or institutions. It provides valuable insights from past data but may require extensive interpretation and verification.
\end{itemize}

Data analysis refers to the process that takes place after data collection, aiming to extract meaningful insights and patterns from the gathered information.

In this thesis, no questionnaires, interviews, observations, or archival research methods were employed, as the focus was strictly on the European research domain.
Given that the dataset used, detailed in the next chapter (Sec.~\ref{sec:dataset}), specifically pertains to EU-funded research projects, the analysis was conducted solely within this structured data framework, ensuring relevance to the scope of European research.