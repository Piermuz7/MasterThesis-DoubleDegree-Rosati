Research philosophy is the first layer of the research onion and refers to the set of beliefs and assumptions that establish the research process.
Following the research onion \cite{SaundersMark2023}, the main philosophies that a researcher can adopt are positivism, critical realism, interpretivism, postmodernism, and pragmatism.

\textit{Positivism} is based on the idea of working with an observable social reality and that the end product of such research can be generalisations similar to those produced by physical and natural scientists.
Since it argues for an objective reality that can be measured by quantitative methods, this thesis does not follow this approach as the main objective is not to generalise the findings.
\textit{Critical realism} is based on the idea that the social world is not a direct reflection of the physical world, but that it is constructed by individuals based on their experiences and perceptions.
This approach is also not followed in this study as it is more focused on understanding the social world as it is perceived by the participants.
\textit{Interpretivism} emphasizes the importance of understanding the differences among human beings in their roles as social actors.
It highlights the distinction between conducting research involving people, who interpret and assign meaning to their experiences, as opposed to research involving inanimate objects.
This thesis does not follow this philosophy because it does not explore subjective human experiences and social constructs.
\textit{Postmodernism} challenges established ideas by emphasizing the role of language, power, and cultural context in shaping reality.
It rejects the notion of a single objective truth, arguing instead for multiple, evolving perspectives influenced by social constructs.
This philosophy is not followed in this thesis as it does not aim to challenge established ideas or explore the role of language, power, and cultural context in shaping reality.
Finally, \textit{pragmatism} focuses on practical solutions, asserting that research should be guided by the problem at hand rather than rigid philosophical stances.
It promotes flexibility by combining qualitative and quantitative methods to achieve useful and actionable outcomes.

This thesis follows a pragmatic approach in that it aims to address the problem of finding collaborators for a given research project.