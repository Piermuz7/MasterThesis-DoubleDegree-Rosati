\abstractita{
Il successo delle proposte di progetti di ricerca dipende in larga misura dal consorzio, che deve avere esperienza e conoscenza dei temi delineati nei relativi bandi, ad esempio quelli dell'ultimo programma di ricerca e innovazione dell'UE Horizon Europe.
Tuttavia, una delle attivit\`a pi\`u impegnative \`e la formazione del consorzio, che richiede l'identificazione di adeguati collaboratori di ricerca.
I metodi tradizionali affrontano questa sfida affidandosi esclusivamente ai social network o al numero di citazioni degli autori, che si sono rivelati di efficacia limitata.
Questa tesi propone un metodo Agentic Graph \Acrfull{rag}, che fornisce raccomandazioni contestuali e spiegabili, adattate alle aree di competenza dei ricercatori e alla rilevanza del progetto, quindi pi\`u efficaci degli approcci esistenti.
Il metodo proposto combina le capacit\`a dei \Acrfullpl{kg} e dei \Acrfullpl{llm} ed \`e stato sviluppato seguendo la metodologia di ricerca della Design Science Research.
Il nuovo metodo \`e stato valutato prendendo in considerazione uno degli \glspl{llm} pi\`u performanti attualmente sul mercato, GPT-4o.
}