\chapter{Introduction}\label{chap:intro}

This chapter presents the foundation of the thesis including the background, challenges, and goals of the study.
It begins with the Background (Sec.~\ref{sec:background}), which provides an overview of research collaboration, its importance in advancing scientific discovery, and the role of digital tools and recommender systems in facilitating such collaborations.
The chapter then moves to the Problem Statement (Sec.~\ref{sec:problem-statement}), highlighting the limitations of traditional approaches and existing \gls{ai} methods in identifying potential research collaborators, such as their lack of interpretability, contextual understanding, and ability to generalize across diverse scenarios.
Sec.~\ref{sec:thesis-statement} defines the Thesis Statement, which outlines the proposed solution: a recommender system integrating knowledge graphs and large language models to deliver accurate, explainable, and context-aware recommendations.
The Research Gap (Sec.~\ref{sec:research-gap}) identifies shortcomings in current methods, emphasizing the unexplored potential of combining knowledge graphs with retrieval-augmented large language models to enhance precision and explainability.
The chapter also illustrates Research Questions (Sec.~\ref{sec:research-questions}) to guide the investigation and concludes with a Thesis Structure (Sec. \ref{sec:thesis-structure}), offering a roadmap for the subsequent chapters of the document.

\section{Background}\label{sec:background}
Research collaboration is the process where researchers from various fields, institutions, or disciplines work together to achieve shared scientific objectives \cite{KATZ19971}.
The exponential growth of academic research in various fields has increased the need for effective collaboration between researchers \cite{Adams2012, Vermeulen2017}.
Research collaborations are commonly used in areas such as medicine, engineering, social sciences, and environmental studies, where diverse expertise is often required to tackle intricate problems.
According to \textcite{Mei2021}, these collaborations are vital for addressing complex, multidisciplinary challenges, advancing knowledge, and accelerating innovation.
They typically take the form of joint publications, shared research grants, co-developed methodologies, and interdisciplinary studies.
Their importance is underscored by their ability to share resources and knowledge and foster innovation, making them key factors in scientific progress.

Traditional approaches to collaboration often rely on professional networking through conferences, workshops, and institutional partnerships \cite{KATZ19971}.
These methods have proven effective in facilitating connections, but are often constrained by geographic and logistical limitations.
Digital platforms, such as ResearchGate\footnote{\url{https://www.researchgate.net}}, Academia.edu\footnote{\url{https://www.academia.edu}}, Google Scholar\footnote{\url{https://scholar.google.com}} and LinkedIn\footnote{\url{https://www.linkedin.com}}, have expanded the reach of collaborations, allowing researchers to network online, share results, and identify potential collaborators around the world.
However, these platforms focus primarily on social connectivity rather than intelligent matchmaking based on complementary expertise or shared goals.

Recommender systems \cite{Lu2012} are \gls{ai} algorithms that make suggestions and recommendations about the most relevant items for a particular user.
Nowadays, they are widely used in commercial contexts such as e-commerce and streaming platforms and have shown significant potential to address the challenge of matching individuals with relevant objects or entities \cite{Hussien2021}.

In recent years, hybrid \gls{ai} has emerged as a novel research field that combines approaches from \gls{ml} and Knowledge Engineering \cite{RordorfKCL23,PraterLaurenzi22,GarcezL23}.
Respectively, two emerging technologies are \glspl{llm} and \glspl{kg}.
These two have very recently been employed for developing intelligent and interpretable recommendation systems \cite{Zhao2024}.
\glspl{kg} enable the representation of structured information about entities and their relationships, providing a basis for reasoning and generating insights.
At the same time, \glspl{llm} have demonstrated remarkable abilities to process and understand complex textual information. 

The Semantic Web, an extension of the World Wide Web for enhancing the current web plays a crucial role in this context.
By embedding structured, machine-readable data into web resources, the Semantic Web allows information to be interconnected and queried across diverse datasets.
Standards such as the \gls{rdf} \cite{Cyganiak14RCA} and \gls{owl} \cite{Deborah2004} form the foundation of the Semantic Web, enabling interoperability and integration of data from various sources.
These capabilities are particularly valuable for applications requiring complex reasoning and decision-making.

The combination of advanced \gls{ai} methodologies with Semantic Web technologies has opened new possibilities for enhancing recommender systems.
By leveraging the structured, interconnected data of the Semantic Web alongside the predictive power of \gls{ai} algorithms, these systems can deliver more accurate, relevant, and context-aware recommendations.
This not only improves the effectiveness of recommender systems but also expands their potential to address the sophisticated requirements of users in academic and scientific collaboration settings.
%
\section{Problem Statement}\label{sec:problem-statement}

Research collaboration is an essential part of modern scientific progress, but it presents many challenges that make it difficult to be effective.
A significant problem comes from disciplinary and cognitive differences: collaborators from different fields often struggle to reconcile conflicting methodologies, priorities, and theoretical perspectives \cite{Bozeman2014}.
Communication barriers further make these challenges more difficult, as insufficient or ineffective exchanges between team members can lead to misunderstandings and a lack of cohesion \cite{Melin1996,Mwantimwa2023}.
Inconsistent levels of commitment within teams often create imbalances, with some researchers prioritizing their personal goals over collective ones, thus straining trust and productivity \cite{Melin1996}.
In addition, disputes over equity, such as work distribution, authorship, and access to resources, can lead to employee dissatisfaction and resentment.
According to \cite{KSubramanyam1983}, ineffective leadership and management practices, such as overregulation or lack of flexibility, often fail to adequately address these problems or may even intensify them.
Cultural and social differences, including variations in organizational norms, gender, or cultural background, while potentially enriching collaboration, can also be challenging if not managed inclusively \cite{ABRAMO20171016}.

Despite their advancements, recommender systems face several limitations that hinder their effectiveness in complex domains such as research collaboration.
Traditional approaches, such as \gls{cbf} and \gls{cf}, often suffer from data sparsity, making it difficult to provide accurate recommendations when limited interaction history is available \cite{Plexousakis2005, WEI201729}.
Cold-start problems further exacerbate this issue, as new users or items lack sufficient data for meaningful suggestions \cite{Lu2012}.
Moreover, many recommender systems struggle with explainability, offering predictions without transparent reasoning, which can reduce user trust and interpretability, particularly in high-stakes decision-making scenarios \cite{AIinRecSys}.
Some recommendation models tend to prioritize historical patterns over new discoveries, limiting their ability to facilitate serendipitous connections between researchers from diverse disciplines \cite{Jagadishwari2023,Iana2021}.
These challenges highlight the need for more sophisticated and knowledge-aware recommender systems that can integrate domain-specific semantics, contextual understanding, and explainable \gls{ai} techniques to improve the quality and fairness of recommendations.
%
\section{Thesis Statement}\label{sec:thesis-statement}

This thesis proposes the design and development of a recommendation system that leverages the power of \glspl{kg} and \glspl{llm} to provide accurate and explainable recommendations for identifying potential research collaborators.
By integrating \glspl{kg}, which provide a structured and interconnected representation of domain-specific information, with the contextual understanding and generative capabilities of \glspl{llm}, the system aims to address key challenges of research collaboration recommendations.
These include understanding complex academic relationships between researchers, research projects and areas of interest of researchers, the forming of research consortia, and explaining the motivations behind recommendations.
The proposed system leverage the strengths of both technologies to retrieve and analyze relevant academic data, dynamically adapt to user preferences, and provide personalized and well-justified recommendations, thus enhancing the process of finding and connecting with suitable research collaborators.
%
\section{Research Gap}\label{sec:research-gap}
Despite the significant advancements in \gls{dl} models for recommender systems, these models exhibit critical limitations in capturing the preferences and diverse textual side information of users.
They often struggle with generalizing to unseen recommendation scenarios and explaining the reasoning behind their predictions \cite{Zhao2024}.
This lack of explainability and adaptability reduces their effectiveness in tasks requiring complex reasoning, such as research collaborator recommendations.

\glspl{llm} have emerged as transformative tools due to their exceptional natural language understanding and generation capabilities, enabling them to understand complex patterns and provide human-like reasoning.
When combined with \gls{rag}, \glspl{llm} can overcome the limitations of \gls{dl} models by integrating external knowledge sources, reducing hallucinations, and providing contextually enriched, explainable, and accurate recommendations \cite{Deldjoo2024}.

This combination represents a promising approach for addressing the challenge of explainability in recommending research collaborators, particularly by aligning recommendations with explicit user queries and leveraging diverse, up-to-date knowledge bases.
However, the \glspl{llm} and \gls{rag}'s integration in this specific context remains underexplored, it is an opportunity to fill this gap and advance the field.
%
\section{Research Questions}\label{sec:research-questions}
The main \gls{rq} is derived from the thesis statement and guides the investigation:
\begin{center}
	\textit{How can a \gls{kg} and \gls{llm}-based approach enhance the process of suggesting collaborators for research projects?}
\end{center}

To address this question, the following subquestions are explored:
\begin{itemize}
	\item \rqOne
	\item \rqTwo
	\item \rqThree
	\item \rqFour
\end{itemize}

%\begin{itemize}
%    \item Knowledge Representation: \textit{how can research-related data (e.g., publications, topics, affiliations) be effectively modeled into a \gls{kg} to capture relationships among researchers?}
%	\item Information Retrieval: \textit{What techniques can be employed to efficiently retrieve relevant information from large, heterogeneous data sources to support personalized recommendations?}
%	\item Explainability: \textit{How can the system generate human-readable explanations for its recommendations using the \gls{kg} and \gls{llm} outputs?}
%	\item Evaluation: \textit{What metrics and evaluation frameworks are suitable to assess the accuracy, usability, and satisfaction of the proposed system?}
%\end{itemize}

By addressing these questions, this thesis aims to contribute to the field of intelligent recommender systems and facilitate impactful research collaborations.

\section{Thesis Structure}\label{sec:thesis-structure}
TODO: Add the structure of the thesis when the chapters are finalized.

