In the problem awareness phase, we addressed the first sub-research question:
\begin{center}
    \rqOne
\end{center}
To establish a foundation for this study, we conducted a comprehensive literature review covering key areas such as research collaboration (Sec.~\ref{sec:research-collaboration}), the role of \glspl{kg} and ontologies in the research domain (Sec.\ref{sec:knowledge-graphs}), \glspl{llm} and their applications (Sec.~\ref{sec:large-language-models}), \gls{rag} aproaches (Sec.~\ref{sec:retrieval-augmented-generation}), recommender systems tailored for research contexts (Sec.~\ref{sec:recommender-systems-in-research-field}), and the emerging field of Retrieval-Augmented Recommender Systems (Sec.~\ref{sec:retrieval-augmented-recommender-systems-in-research-field}).
From the literature review conducted, existing \glspl{kg} in the research field, such as \gls{mag}, \gls{s2ag}, Wikidata, \gls{orkg}, and VIVO ontology, were analyzed.
However, they present several limitations that impact their effectiveness in research collaborator recommendation such as:
\begin{itemize}
    \item Lack of granular and up-to-date data: many of these \glspl{kg} rely on static datasets or periodic updates, making it challenging to capture evolving research trends, emerging collaborations, and newly published work in real time.
    \item Limited contextual understanding: while these \glspl{kg} store structured information about authors, affiliations, and citations, they often lack deeper semantic connections, such as researchers' interdisciplinary expertise, project alignments, or implicit collaboration potential.
    \item Explainability and interpretability issues: many existing \glspl{kg} focus on metadata-level relationships rather than explaining why certain collaborations might be beneficial. This limits their ability to support explainable \gls{ai}-driven recommendations.
    \item Integration challenges: despite leveraging \gls{lod} principles, these \glspl{kg} are often designed with distinct schemas and ontologies, leading to interoperability challenges when combining multiple sources for enriched recommendations.
\end{itemize}

Given these limitations, the \gls{eurio} \gls{kg} was identified as particularly suitable for modeling European projects, making it an excellent choice for this thesis.
\gls{eurio} provides structured, rich metadata on European-funded projects, including participants, topics, and collaborations.
By leveraging \gls{eurio}, this thesis aims to overcome the limitations of existing \glspl{kg}, enabling a more dynamic, explainable, and semantically enriched recommendation framework.
The \gls{eurio} ontology and \gls{kg} are described in the following subsections.

Moreover, we found that most of the papers analyzed on recommender systems in the research domain, are designed to suggest research papers or to recommend collaborators.
In both cases, the most relevant aspects on which the recommender algorithms are based are influences on the number of citations, and abstracts of the papers.
Despite their advancements, recommender systems face significant limitations, particularly in complex domains like research collaboration.
Traditional methods such as \gls{cbf} and \gls{cf} suffer from data sparsity, making accurate recommendations difficult when user interaction data is limited.
The cold-start problem further complicates this by preventing effective suggestions for new users or items.
Additionally, many systems lack explainability, reducing user trust and interpretability, especially in critical decision-making contexts.
Another challenge is the tendency to prioritize historical patterns over novel discoveries, restricting interdisciplinary connections.
These limitations underscore the need for knowledge-aware recommender systems that integrate domain-specific semantics, contextual understanding, and explainable \gls{ai} techniques to enhance recommendation quality, fairness, and effectiveness.

We will focus on the identified research gap, which is in the limitations of \gls{dl}-based recommendation systems in capturing users' preferences, handling diverse textual information, and providing explainable recommendations.
These models often struggle to generalize to unseen scenarios and lack interpretability, making them less effective for tasks that require complex reasoning, such as recommendations from research collaborators.

\subsection*{EURIO: EUropean Research Information Ontology}
The \gls{eurio} ontology, developed by the Publications Office of the European Union\footnote{\url{https://op.europa.eu/en/}}, is a data model which conceptualizes, formally encodes, and makes available in an open, structured, and machine-readable format data about research projects funded by the EU's framework programmes for research and innovation.
\gls{cordis}, as described in Sec.~\ref{sec:dataset}, is responsible for publishing the results of these projects, while \gls{eurio} provides a semantic model that enhances transparency, reusability, and accessibility.
The \gls{eurio} ontology is built on top of well-known ontologies and vocabularies to ensure interoperability and semantic richness.
These include:
\begin{itemize}
    \item \gls{dc}: used for metadata elements such as titles, descriptions, and identifiers.
    \item \gls{dcat}: an \gls{rdf} vocabulary designed to facilitate interoperability between data catalogs published on the Web.
    \item \gls{dingo}: an ontology expressly designed to provide an extensible interoperable framework for formally conceptualizing and expressing the relevant parts of the research/cultural landscape in relation to funding, such that they can easily be shared between different actors and platforms.
    \item \gls{fabio}: facilitates the description of bibliographic entities and their relationships.
    \item \gls{frapo}: an ontology for describing the administrative information of research projects, e.g., grant applications, funding bodies, project partners, etc.
    \item \gls{foaf}: defines relationships between people and organizations.
    \item \gls{skos}: facilitates controlled vocabularies and classification schemes.
\end{itemize}

The \gls{eurio} ontology also incorporates reference data, such as countries, funding schemes, types of action, the EuroSciVoc taxonomy, and the NUTS classification, to enhance the semantic representation of research information.
It leverages the \gls{owl} 2 to formally define the semantics of domain-specific terms used to describe \gls{cordis} entities (e.g., projects, organizations, etc.), their attributes (e.g., title, acronym, legal name, etc.), and their interrelations (e.g., the connection between a project and its participating organizations, etc.).

The \gls{eurio} ontology defines multiple classes representing different concepts such as projects, organizations, funding schemes, grants, publications, and roles, along with associated data properties and object properties that define their relationships.
Each class has a set of data properties that describe its attributes, and a set of object properties that define its connections to other classes.
Each class, data property, and object property has several annotations but in general the main annotations for those are \textit{rdfs:label}, which provides a human-readable label for the entity, \textit{rdfs:comment}, which provides a human-readable description of the entity, and \textit{rdfs:isDefinedBy}, which provides a link to the ontology where the entity is defined.
For example, the title, description, start date, end date, and funding amount are data properties of the \textbf{Project} class.

\begin{figure}[htbp]
    \centering
 \includegraphics[width=.9\textwidth]{figures/architecture/is-funded-by.png}
     \rule{35em}{0.5pt}
    \caption{The ``is funded by'' relationship between a project and its grant}
 \label{fig:is-funded-by}
\end{figure}

The ontology also defines relationships between classes, such as \textbf{is funded by}, the association between a project and its grant (Fig.~\ref{fig:is-funded-by}).
Another example or relationship is \textbf{is employed by}, which links a person, in particular a \textbf{Person Role} to an \text{Organisation}.
The ontology also provides object property descrptions, for example the \textbf{has involved party} relationship, which links a \textbf{Project} to a \textbf{Role}, is inverse of \textbf{is involved in}, which links a \textbf{Role} to a \textbf{Project} (Fig.~\ref{fig:is-involved-in-has-involved-party}).

\begin{figure}[htbp]
    \centering
 \includegraphics[width=.9\textwidth]{figures/architecture/has-involved-party-is-involved-in.png}
     \rule{35em}{0.5pt}
    \caption{The ``has involved party'' and ``is involved in'' relationships}
 \label{fig:is-involved-in-has-involved-party}
\end{figure}

This object property descriptions can be very useful for querying the \gls{kg} to extract relevant information in a structured manner.
Moreover, the ontology defines a hierarchy of classes.
For example, the \textbf{Organisation} class is a superclass of several subclasses such as \textbf{For Profit Organisation}, \textbf{Funding Agency}, \textbf{Higher Or Secondary Education}, \textbf{Research Organisation}, and \textbf{SME} (Small and Medium Enterprise) (Fig.~\ref{fig:organisation-class-hierarchy}).

\begin{figure}[htbp]
    \centering
 \includegraphics[width=.9\textwidth]{figures/architecture/organisation-class-hierarchy.png}
     \rule{35em}{0.5pt}
    \caption{The ``Organisation'' class hierarchy}
 \label{fig:organisation-class-hierarchy}
\end{figure}

An overview of the \gls{eurio} ontology is shown in the UML diagram in Fig.~\ref{fig:eurio-ontology}.

\begin{figure}[htbp]
    \centering
 \includegraphics[width=\textwidth]{figures/architecture/EURIO_V2.4.png}
     \rule{35em}{0.5pt}
    \caption{A graphical representation of the \gls{eurio} ontology (from \url{https://op.europa.eu/en/web/eu-vocabularies/eurio})}
 \label{fig:eurio-ontology}
\end{figure}

\subsection*{Data Availability and Formats}
The dataset is structured in \gls{rdf} format and contains over 18 million \gls{rdf} triples, with a total size of approximately 5GB when serialized in the N-Quad Triples \gls{rdf} format.
These structured triples enable the representation of relationships between projects, organizations, researchers, project publications, funding schemes, grants, countries, and other relevant entities making, the dataset a rich resource for \gls{kg}-based research.

The \gls{eurio} \gls{kg} is available in multiple formats to ensure accessibility and ease of integration into different research workflows.
Supported formats include \gls{rdf}, \gls{ttl}, \gls{nq}, \gls{jsonld}, and \gls{nt}.
The \gls{eurio} \gls{kg} used in this thesis is the latest version updated by the European Union on 08.11.2023, ensuring that the data remains relevant and accurate up to this date.

\section*{Knowledge Graph Exploration}
As previously explained, the \gls{eurio} \acrlong{kg} \cite{CORDIS_EURIO_2022} is built upon \gls{cordis} data and serves as a structured \gls{kg} that encapsulates information about research projects funded under the \gls{fp7} and \gls{h2020} framework programmes.
The \gls{eurio} \gls{kg} can be accessed via a \gls{sparql} endpoint at \url{https://cordis.europa.eu/datalab/sparql-endpoint/en}.
The \gls{kg} provides both database dumps and subsets of \gls{eurio} data in the form of named graphs.
The structure and organization of these named graphs are defined by the \gls{eurio} ontology, which is publicly available at: \url{https://op.europa.eu/en/web/eu-vocabularies/eurio}.

As part of this study, we conducted an exploration of the \gls{eurio} \gls{kg}, focusing on analyzing and understanding its ontology schema by identifying and examining the most important concepts, relationships, and structural elements.
This initial exploration allowed us to gain insights into the way projects, organizations, researchers, and related entities are modeled within the dataset.
For example, the Fachhochschule Nordwestschweiz and University of Camerino  subgraphs were extracted from the \gls{eurio} \gls{kg} to understand their properties and relationships.
Figures~\ref{fig:fhnw-graphdb} and~\ref{fig:unicam-graphdb} show the subgraphs for Fachhochschule Nordwestschweiz and University of Camerino, respectively, highlighting the connections between other entities.

\begin{figure}[htbp]
    \centering
 \includegraphics[width=.9\textwidth]{figures/architecture/graphdb-fhnw.png}
     \rule{35em}{0.5pt}
    \caption{FACHHOCHSCHULE NORDWESTSCHWEIZ subgraph in the \gls{eurio} \gls{kg}}
 \label{fig:fhnw-graphdb}
\end{figure}

\begin{figure}[htbp]
    \centering
 \includegraphics[width=.9\textwidth]{figures/architecture/graphdb-unicam.png}
     \rule{35em}{0.5pt}
    \caption{University of Camerino subgraph in the \gls{eurio} \gls{kg}}
 \label{fig:unicam-graphdb}
\end{figure}

As can be seen from both figures, both the Fachhochschule Nordwestschweiz and the University of Camerino are instances of the \textbf{eurio:HigherOrSecondaryEducation} class and have the following six data properties: \textit{rdfs:label}, \textit{eurio:url}, \textit{eurio:vatNumber}, \textit{eurio:legalName}, \textit{eurio:rcn}, and \textit{eurio:identifier}.
Furthermore, we can see that, taking Fachhochschule Nordwestschweiz as an example, it has several \textit{eurio:hasRole} relationships with different ``Org: FACHHOCHSCHULE NORDWESTSCHWEIZ'' entities.
The \gls{kg} contains several such redundancies, also for other types of classes.
Each entity of type \textbf{eurio:Organisation} has several relationships with the entity of type \textbf{eurio:OrganisationRole}.
This type of relationship implies a lot of redundancy in the nodes of the \gls{kg}, but is deduced to be necessary to represent the project-related organisation role.
For example, as illustrated in Fig.~\ref{fig:fhnw-organisationRole-example}, the Fachhochschule Nordwestschweiz node is linked via the \textit{euro:hasRole} relationship, to the node ``Org: FACHHOCHSCHULE NORDWESTSCHWEIZ/ Role: participant/ Project: 53135'', which semantically represents the participation of the Fachhochschule Nordwestschweiz in the project with project ID 53135.
From the data properties of this node, we can easily see that the role of the FHNW in this project was participant, and that the project duration was from April 4, 2018, to September 30, 2021.

\begin{figure}[htbp]
    \centering
 \includegraphics[width=.9\textwidth]{figures/architecture/graphdb-fhnw-organisationRole-example.png}
     \rule{35em}{0.5pt}
    \caption{An example of an OrganisationRole entity in the FHNW subgraph}
 \label{fig:fhnw-organisationRole-example}
\end{figure}

Following this, we performed a more in-depth exploration using \gls{sparql} queries to retrieve and analyze various aspects of the \gls{kg}.
This included examining the metadata of research projects, their abstract, start and end dates, project status, as well as the organizations involved, their roles, and the researchers affiliated with them.
For example, Listing~\ref{lst:sparql_example_project1_data_properties} shows a \gls{sparql} query that retrieves the abstract, status, start date, and end date of a specific project titled ``BIM-based holistic tools for Energy-driven Renovation of existing Residences'', which was selected as a scenario example (Project \ref{project1}) for this study in Sec.~\ref{sec:scenarios}.
The results of this query are shown in Fig.~\ref{fig:sparql_example_project1_data_properties}, providing detailed information about some project data properties.

\begin{lstlisting}[language=SPARQL, caption={\gls{sparql} query for getting the abstract, status, start date, and end date of a project titled ``BIM-based holistic tools for Energy-driven Renovation of existing Residences''}, label=lst:sparql_example_project1_data_properties]
    PREFIX eurio: <http://data.europa.eu/s66#>
    SELECT ?abstract ?status ?start_date ?end_date
    WHERE {
        ?project a eurio:Project ; 
            eurio:title "BIM-based holistic tools for Energy-driven Renovation of existing Residences" ;
            eurio:abstract ?abstract ;
            eurio:startDate ?start_date ;
            eurio:endDate ?end_date ;
            eurio:projectStatus ?status .
    }
\end{lstlisting}

\begin{figure}[htbp]
    \centering
 \includegraphics[width=.8\textwidth]{figures/architecture/sparql_example_project1_data_properties.png}
     \rule{35em}{0.5pt}
    \caption{\gls{sparql} query results for getting the abstract, status, start date, and end date of a project titled ``BIM-based holistic tools for Energy-driven Renovation of existing Residences''}
 \label{fig:sparql_example_project1_data_properties}
\end{figure}

We also explored the consortium participants involved in the selected project by querying the \gls{kg} to retrieve the organizations and their roles in the project.
Listing~\ref{lst:sparql_example_project1_consortium_participants} shows the \gls{sparql} query used to extract the consortium participants involved in the project \ref{project1}.
The results of this query are shown in Fig.~\ref{fig:sparql_example_project1_consortium_participants}, providing detailed information about the organizations and their roles in the project.
As we can see, the consortium participants are the same as the ones listed in the scenario analysis in Sec.~\ref{sec:scenarios}.

\begin{lstlisting}[language=SPARQL, caption={\gls{sparql} query for getting the consortium participants involved in a project titled ``BIM-based holistic tools for Energy-driven Renovation of existing Residences''}, label=lst:sparql_example_project1_consortium_participants]
    PREFIX eurio: <http://data.europa.eu/s66#>
    PREFIX rdfs: <http://www.w3.org/2000/01/rdf-schema#>
    SELECT DISTINCT (COALESCE(?org, STRBEFORE(STRAFTER(?party_title, "Org: "), "/ Role:")) AS ?organisation) ?role_label
    WHERE {
        ?project a eurio:Project .
        ?project eurio:title "BIM-based holistic tools for Energy-driven Renovation of existing Residences".
        ?project eurio:title ?project_title.
        ?project eurio:hasInvolvedParty ?party .
        ?party a eurio:OrganisationRole.
        ?party eurio:roleLabel ?role_label.
        ?party rdfs:label ?party_title.
        OPTIONAL { ?party eurio:isRoleOf ?role . ?role rdfs:label ?org. }
    }
\end{lstlisting}

\begin{figure}[htbp]
    \centering
 \includegraphics[width=.8\textwidth]{figures/architecture/sparql_example_project1_consortium_participants.png}
     \rule{35em}{0.5pt}
    \caption{\gls{sparql} query results for getting the consortium participants involved in a project titled ``BIM-based holistic tools for Energy-driven Renovation of existing Residences''}
 \label{fig:sparql_example_project1_consortium_participants}
\end{figure}

During the exploration, we also investigated people involved in research projects.
In the \gls{eurio} \gls{kg}, every project includes information about the organizations involved.
Unfortunately, not all projects have the \textit{eurio:isEmployedBy} relationship linking a \textbf{eurio:Role} to a \textbf{eurio:Organisation}.
As a result, it is not always possible to retrieve details about the individuals associated with a given project, since their employment or involvement is not explicitly represented in all cases.
The selected project \ref{project1}, is an example of a project where the people involved are not explicitly linked to the project.
Running the query in Listing~\ref{lst:sparql_example_project1_people} on the \gls{eurio} \gls{kg} did not return any results, indicating that the people involved in the project are not explicitly linked to the project.

\begin{lstlisting}[language=SPARQL, caption={\gls{sparql} query for getting full name, organisation, telephone, and fax of the people involved in a project titled ``BIM-based holistic tools for Energy-driven Renovation of existing Residences''}, label=lst:sparql_example_project1_people]
    PREFIX eurio: <http://data.europa.eu/s66#>
    PREFIX rdfs: <http://www.w3.org/2000/01/rdf-schema#>
    SELECT ?person_label ?org_name ?telephone ?fax
    WHERE {
        ?person a eurio:Person .
        ?person rdfs:label ?person_label .
        ?person eurio:hasContactDetails ?contact_details.
        ?contact_details eurio:telephone ?telephone.
        ?contact_details eurio:faxNumber ?fax.
        ?person eurio:hasRole ?role.
        ?role eurio:isEmployedBy ?org.
        ?org rdfs:label ?org_name.
        ?role eurio:isInvolvedIn ?project.
        ?project a eurio:Project.
        ?project eurio:title "BIM-based holistic tools for Energy-driven Renovation of existing Residences".
    }
\end{lstlisting}

Listing~\ref{lst:sparql_example_merelli_project_people} is the same query as Listing~\ref{lst:sparql_example_project1_people}, but for a different project titled ``Topology driven methods for complex systems''.
The results of this query are shown in Fig.~\ref{fig:sparql_example_merelli_project_people}, providing detailed information about the people involved in the project, including their full name, organization, telephone, and fax.

\begin{lstlisting}[language=SPARQL, caption={\gls{sparql} query for getting full name, organisation, telephone, and fax of the people involved in a project titled ``Topology driven methods for complex systems''}, label=lst:sparql_example_merelli_project_people]
    PREFIX eurio: <http://data.europa.eu/s66#>
    PREFIX rdfs: <http://www.w3.org/2000/01/rdf-schema#>
    SELECT ?person_label ?org_name ?telephone ?fax
    WHERE {
        ?person a eurio:Person .
        ?person rdfs:label ?person_label .
        ?person eurio:hasContactDetails ?contact_details.
        ?contact_details eurio:telephone ?telephone.
        ?contact_details eurio:faxNumber ?fax.
        ?person eurio:hasRole ?role.
        ?role eurio:isEmployedBy ?org.
        ?org rdfs:label ?org_name.
        ?role eurio:isInvolvedIn ?project.
        ?project a eurio:Project.
        ?project eurio:title "Topology driven methods for complex systems".}
\end{lstlisting}

\begin{figure}[htbp]
    \centering
 \includegraphics[width=.9\textwidth]{figures/architecture/sparql_example_merelli_project_people.png}
     \rule{35em}{0.5pt}
    \caption{\gls{sparql} query results for getting full name, organisation, telephone, and fax of the people involved in a project titled ``Topology driven methods for complex systems''}
 \label{fig:sparql_example_merelli_project_people}
\end{figure}

The last person in the results of the query in Listing~\ref{lst:sparql_example_merelli_project_people} is Prof. Emanuela Merelli, who is affiliated with the University of Camerino, and Fig.~\ref{fig:example-person-prof-merelli} presents a visual representation of the \gls{eurio} \gls{kg}, specifically focusing on her class type as a \textbf{eurio:Person}.

\begin{figure}[htbp]
    \centering
 \includegraphics[width=.9\textwidth]{figures/architecture/example-person-prof-merelli.png}
     \rule{35em}{0.5pt}
    \caption{An example of a person's profile in the \gls{eurio} \gls{kg}}
 \label{fig:example-person-prof-merelli}
\end{figure}

At the center of the graph, her node connects to multiple entities representing affiliations, roles, projects, and contact details.
The right panel provides metadata about her, including her given name, family name, research control number, and honorific title, indicating her academic status as ``Prof.''.
From Emanuela Merelli's node, one connection leads to a contact details node, which contains her telephone number (+390737402567) and fax number (+390737402561), establishing the \textit{eurio:hasContactDetails} relationship.
Another key connection links her to the ``Universit\`a degli Studi di Camerino'', represented as a green node, with the \textit{eurio:isEmployedBy} relationship, signifying her affiliation with this institution.
Her professional involvement is further detailed through an orange node representing her role, which is bidirectionally linked to her node through the \textit{eurio:hasRole} and \textit{eurio:isRoleOf} relationships.
This \textbf{eurio:PersonRole} node serves as an intermediary between her and the projects she is involved in.
Another significant connection in the graph extends from the role node to a purple node, which is an instance of \textbf{eurio:Project} class, and represents a research project titled ``Topology driven methods for complex systems'', indicating her involvement.
The connection between the role and the project is represented by the \textit{eurio:isInvolvedIn} relationship, confirming her participation in the research activity.
Additional purple nodes represent the classification of roles, distinguishing between general role types and specific person roles.


Through this query-based investigation, we were able to extract detailed information on how projects are interconnected, how participants are structured, and how different elements contribute to the overall knowledge representation within \gls{eurio}.
This \gls{kg} exploration provided a comprehensive understanding of the dataset's richness and limitations, enabling a more informed approach in leveraging the \gls{kg} for our study.

Leveraging the \gls{eurio} ontology, we can effectively address the first sub-research question by structuring and representing data on researchers and research projects within a \gls{kg}.