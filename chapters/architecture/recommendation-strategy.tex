

The recommendation strategy in this study was designed to leverage the structured data and relationships within the \gls{eurio} \gls{kg} to generate meaningful and context-aware recommendations.
Given the rich metadata and interconnected entities in \gls{eurio}, we exploited its data properties and relationships to extract the most relevant information about researchers, organizations, and projects.
To facilitate the efficient retrieval of information, we developed a mechanism to query and extract the most meaningful and useful relationships for our purposes, enabling the identification of essential project details, such as the participants in a project, a person's employing organisation, and the organisation's role in a project.
Other useful information to be retrieved is the abstract, duration, status, and url of a project given its title.


\subsection*{Types of Recommendations}
Based on this structured data retrieval, we designed the two following primary types of recommendations.
By integrating these structured recommendations, this strategy enhances the discoverability of research partnerships and collaboration opportunities, leveraging the semantic richness of the \gls{eurio} \gls{kg} to provide personalized and explainable suggestions. The approach ensures that both individual researchers and research organizations receive tailored recommendations based on contextual relevance and domain-specific alignment, making the system an effective tool for fostering research collaboration.

\subsubsection*{Research Collaborator Recommendations}
The first approach focuses on recommending potential research collaborators based on a given project description and its objectives.
By analysing the key attributes of a project, including the title, description and research objectives, the system identifies researchers with expertise in similar fields, ensuring that recommended collaborators are in line with the project's needs, and that they can contribute through their expertise in one or more specific areas, useful for achieving the project's objectives.
In this type of recommendation, it is also useful to refer to projects similar to the one given as input, in which the suggested individual has been involved.

\subsubsection*{Research Consortium Recommendations}
The second approach aims to suggest potential organizations suitable for forming a research consortium, given a project description and objectives.
This recommendation process considers organizational expertise, prior involvement in similar research initiatives, and institutional capabilities, ensuring that the suggested consortium members complement the project's goals.
This type of recommendation is similar to the first one, but here unlike suggesting a list of researchers who are suitable to collaborate on the project, one or more consortia are suggested.
A suggested consortium consists of a group of organisations, which, on the basis of their past participation in other projects, have experience and expertise in achieving one or more of the project objectives.