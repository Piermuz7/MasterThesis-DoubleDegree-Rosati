\textbf{TODO: add requirements table from the problem}

In the Problem Awareness phase, we addressed the first sub-research question:
\begin{center}
    \rqOne
\end{center}
To establish a foundation for this study, we conducted a comprehensive literature review covering key areas such as research collaboration (Sec.~\ref{sec:research-collaboration}), the role of \glspl{kg} and ontologies in the research domain (Sec.\ref{sec:knowledge-graphs}), recommender systems tailored for research contexts (Sec.~\ref{sec:recommender-systems-in-research-field}), and the emerging field of Retrieval-Augmented Recommender Systems (Sec.~\ref{sec:retrieval-augmented-recommender-systems-in-research-field}).
From the literature review conducted, existing knowledge graphs in the research field, such as \gls{mag}, \gls{s2ag}, Wikidata, \gls{orkg}, and VIVO ontology, were analyzed.
However, they present several limitations that impact their effectiveness in research collaborator recommendation such as:
\begin{itemize}
    \item Lack of granular and up-to-date data: many of these \glspl{kg} rely on static datasets or periodic updates, making it challenging to capture evolving research trends, emerging collaborations, and newly published work in real time.
    \item Limited contextual understanding: while these \glspl{kg} store structured information about authors, affiliations, and citations, they often lack deeper semantic connections, such as researchers' interdisciplinary expertise, project alignments, or implicit collaboration potential.
    \item Explainability and interpretability issues: many existing \glspl{kg} focus on metadata-level relationships rather than explaining why certain collaborations might be beneficial. This limits their ability to support explainable \gls{ai}-driven recommendations.
    \item Integration challenges: despite leveraging \gls{lod} principles, these \glspl{kg} are often designed with distinct schemas and ontologies, leading to interoperability challenges when combining multiple sources for enriched recommendations.
\end{itemize}

Given these limitations, the \gls{eurio} \gls{kg} was identified as particularly suitable for modeling European projects, making it an excellent choice for this thesis.
\gls{eurio} provides structured, rich metadata on European-funded projects, including participants, topics, and collaborations.
By leveraging \gls{eurio}, this thesis aims to overcome the shortcomings of existing \glspl{kg}, enabling a more dynamic, explainable, and semantically enriched recommendation framework.

Moreover, we found that currently there are only recommender systems for suggesting research papers, based on the number of citations, abstracts, keywords etc.
Most of the papers analyzed on recommender systems in the research domain, use \glspl{dnn} to suggest research papers or to recommend collaborators.
In both cases, the most relevant aspects on which the recommender algorithms are based are influences on the number of citations, abstracts of the papers.
In addition, the graphs used in such neural networks, are static, they do not change over time, and it is a problem since in the real world, researchers submit scientific publications constantly, and capturing new relationships between authors, new topics to consider etc is very important in order to suggest accurate recommendations.

We will focus on the identified research gap, which is in the limitations of \gls{dl}-based recommendation systems in capturing users' preferences, handling diverse textual information, and providing explainable recommendations.
These models often struggle to generalize to unseen scenarios and lack interpretability, making them less effective for tasks that require complex reasoning, such as recommendations from research collaborators.