In this study, the \gls{cordis} dataset \cite{CORDIS_FP7_2015,CORDIS_H2020_2015,CORDIS_RefData_2018} was chosen as the primary data source.
\gls{cordis}\footnote{\url{https://cordis.europa.eu}} serves as the European Commission's primary public repository for distributing information about EU-funded research projects.
This dataset is a critical resource for analyzing research trends, understanding research project collaborations, and identifying potential research partners.
The chosen dataset consists of \gls{rdf} representations of projects funded under two major European Union research initiatives:
\begin{itemize}
    \item The \gls{fp7} for Research and Technological Development\footnote{\url{https://cordis.europa.eu/programme/id/FP7}}, covering projects funded from 2007 to 2013.
    \item The \gls{h2020} Programme for Research and Innovation\footnote{\url{https://cordis.europa.eu/programme/id/H2020}}, covering projects funded from 2014 to 2020.
\end{itemize}

The dataset is structured in \gls{rdf} format and contains over 18 million \gls{rdf} triples, with a total size of approximately 5GB when serialized in the N-Quad Triples \gls{rdf} format.
These structured triples enable the representation of relationships between projects, organizations, researchers, project publications, funding schemes, grants, countries, and other relevant entities making, the dataset a rich resource for \gls{kg}-based research.

TODO: add detailed datatset structure, lacks, limitations and so on